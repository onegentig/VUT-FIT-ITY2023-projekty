\documentclass[a4paper, 11pt]{article}

	\usepackage[utf8]{inputenc}
	\usepackage[IL2]{fontenc}
	\usepackage[top=3cm, left=2cm, text={17cm, 24cm}]{geometry}
	\usepackage[backend=bibtex,style=iso-authoryear]{biblatex}
	\usepackage[czech]{babel}
	\usepackage[unicode, hidelinks]{hyperref}
	\usepackage{breakurl}
	\usepackage{csquotes}

	\addbibresource{proj4.bib}

\begin{document}

\begin{titlepage}
	\begin{center}
		{\Huge \textsc{Vysoké učení technické v~Brně}\\}
		{\huge \textsc{Fakulta informačních technologií}\\}
		\vspace{\stretch{0.382}}
		{\LARGE Typografie a~publikování\,--\,4.\ projekt\\}
		{\Huge Bibliografické citace\\}
		\vspace{\stretch{0.618}}
		{\Large 2023 \hfill Onegen\,Niekto}
	\end{center}
\end{titlepage}

\begin{center}
	\Large
	\textbf{Důležitost typografie a kaligrafie v~estetice dokumentů}
	\bigskip
\end{center}
\section*{Význam typografie}

Typografie je silný a~důležitý nástroj v~komunikaci a~designu, který utváří
způsob, jakým čtenář přijme informace v~textu.
Jak uvádí~\textcite{Koch:2012:EmotionIT}: \uv{\emph{Grafický design
		a~typografie hraje důležitou roli ve~snaze pomoci lidem
		dešifrovat významy, stanovit priority informací a~posoudit
		osobní význam sdělení tím, že do~vizuálních sdělení vnáší emoce.}}
Je prokázáno, že typografie a~písmo nevytvářejí pouze estetický dojem,
ale jsou samy o~sobě prostředkem komunikace. Stejně jako volba obrázku
nebo~formulace vytváří typografie spolu s~obsahem
celkové sdělení zprávy. \parencite{Pilka:2019:CharakterPisma}
Jak to? V~roce 1993 provedli Poffenberger a~Barrows experiment,
ve~kterém dokázali, že jednoduché čáry mohou v~člověku vyvolat určité emoce.
Pro~mozek je pohyb oka při sledování linie podobný fyzickému pohybu,
což evokuje samotnou emoci, např.~klesající linie evokují smutek nebo~slabost,
zatímco stoupající linie jsou spojeny s~hravostí nebo~radostí.
\parencite{Sladovnikova:2021:EmoceTextu}

\newpage
\renewcommand{\refname}{Literatúra}
\printbibliography{}

\end{document}