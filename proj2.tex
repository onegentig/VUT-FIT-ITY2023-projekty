\documentclass[a4paper, twocolumn, 11pt]{article}

	\usepackage[top=2.3cm, left=1.4cm, text={18.2cm, 25.2cm}]{geometry}
	\usepackage{amsthm, amsmath, amssymb}
	\usepackage[utf8]{inputenc}
	\usepackage[czech]{babel}
	\usepackage[IL2]{fontenc}
	\usepackage{times}

\begin{document}

\begin{titlepage}
	\begin{center}
		{\Huge \textsc{Vysoké učení technické v~Brně}\\[0.5em]}
		{\huge \textsc{Fakulta informačních technologií}}\\
		\vspace{\stretch{0.382}}
		{\LARGE Typografie a publikování\,--\,2.\ projekt\\[0.4em]
			Sazba dokumentů a matematických výrazů}\\
		\vspace{\stretch{0.618}}
	\end{center}
	{\Large 2023 \hfill Onegen Niekto (xonege99)}
\end{titlepage}

\section*{Úvod}

V~této úloze si vyzkoušíme sazbu titulní strany, mate\-matických vzorců,
prostředí a dalších textových struktur obvyklých pro technicky zaměřené texty\,--\,například
Defi\-nice 1 nebo rovnice (3) na straně 1. Pro vytvoření těchto odkazů používáme kombinace příkazů
\verb|\label|, \verb|\ref|, \verb|\eqref| a \verb|\pageref|. Před odkazy patří nezlomitelná mezera.
Pro zvýrazňování textu jsou zde několikrát použity příkazy \verb|\verb| a \verb|\emph|.

Na titulní straně je použito prostředí \texttt{titlepage} a sázení nadpisu podle
optického středu s~využitím \emph{přesného} zlatého řezu. Tento postup byl probírán na přednášce.
Dále jsou na titulní straně použity čtyři různé velikosti písma a mezi dvojicemi řádků textu
je použito odřádkování se zadanou relativní velikostí 0,5\,em a
0,4\,em\footnote{Nezapomeňte použít správný typ mezery mezi číslem a jednotkou.}.

\section{Matematický text}

\subsection{Podsekce obsahující definici a větu}

\section{Rovnice}

\section{Matice}

\end{document}