\documentclass[a4paper, twocolumn, 10pt]{article}

    \usepackage[top=1.5cm, left=1.6cm, text={18cm, 24cm}]{geometry}
    \usepackage[unicode, hidelinks]{hyperref}
    \usepackage[utf8]{inputenc}
    \usepackage[czech]{babel}
    \usepackage[IL2]{fontenc}
    \usepackage{color}

\title{Typografie a publikování\,–-\,1.\ projekt}
\author{Onegen\,Niěkdo \\
    \href{mailto:xonege99@vutbr.cz}{xonege99@vutbr.cz}}
\date{}

\hyphenation{od-stavcích}

\begin{document}

\maketitle

\section{Hladká sazba}

Hladká sazba používá jeden stupeň, druh a řez písma.
Sází se na stránku s~pevně stanovenou šířkou.
Skládá se z~odstavců. Odstavec končí východovou řádkou.
Věty nesmějí začínat číslicí. \par
Zvýraznění barvou, podtržením, ani změnou písma se v~odstavcích nepoužívá.
Hladká sazba je určena především pro delší texty, jako je beletrie.
Porušení konzistence sazby působí v~textu rušivě a unavuje čtenářův zrak.

\section{Smíšená sazba}

Smíšená sazba má o~něco volnější pravidla. Klasická hladká
sazba se doplňuje o~další řezy písma~pro zvýraznění důležitých pojmů.
Existuje \uv{pravidlo}:

\begin{quotation}

	Čím více \texttt{druhů}, \textit{řezů}, {\tiny velikostí}, \textcolor{green}{barev} písma
	a~{\scshape\color{blue} jiných efektů} \underline{použijeme}, \textcolor{red}{tím profesionál\-něji}
	bude {\fontfamily{pzc}\selectfont dokument} vypadat. Čtenář tím bude
		{\Huge \textbf{vždy nadšen!}}

\end{quotation}

{\scshape Tímto pravidlem se \underline{nesmíte nikdy řídit.}}
Příliš časté zvýrazňování textových elementů a změny {\scriptsize velikosti} písma
jsou známkou \textbf{amatérismu autora} a působí \texttt{velmi rušivě}.
Dobře navržený dokument nemá obsahovat více než 4\,řezy či druhy písma.
Dobře navržený dokument je \underline{decentní, ne chaotický}.

\end{document}
