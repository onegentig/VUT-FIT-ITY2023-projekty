\documentclass[a4paper, 11pt]{article}

	\usepackage[utf8]{inputenc}
	\usepackage[IL2]{fontenc}
	\usepackage[top=3cm, left=2cm, text={17cm, 24cm}]{geometry}
	\usepackage[backend=bibtex,style=iso-authoryear]{biblatex}
	\usepackage[czech]{babel}
	\usepackage[unicode, hidelinks]{hyperref}
	\usepackage[nodayofweek]{datetime}
	\usepackage[hyphenbreaks]{breakurl}
	\usepackage{csquotes}

	\addbibresource{proj4.bib}

\begin{document}

\begin{titlepage}
	\begin{center}
		{\Huge \textsc{Vysoké učení technické v~Brně}\\}
		{\huge \textsc{Fakulta informačních technologií}\\}
		\vspace{\stretch{0.382}}
		{\LARGE Typografie a~publikování\,--\,4.\ projekt\\}
		{\Huge Bibliografické citace\\}
		\vspace{\stretch{0.618}}
		{\Large 2023 \hfill Onegen\,Niekto}
	\end{center}
\end{titlepage}

\begin{center}
	\Large
	\textbf{Důležitost typografie a kaligrafie v~estetice dokumentů}
	\bigskip
\end{center}
\section*{Typografie a~emoce}

Typografie je silný a~důležitý nástroj v~komunikaci a~designu, který utváří
způsob, jakým čtenář přijme informace v~textu.
Jak uvádí~\textcite{Koch:2012:EmotionIT}: \uv{\emph{Grafický design
		a~typografie hraje důležitou roli ve~snaze pomoci lidem
		dešifrovat významy, stanovit priority informací a~posoudit
		osobní význam sdělení tím, že do~vizuálních sdělení vnáší emoce.}}
Je prokázáno, že typografie a~písmo nevytvářejí pouze estetický dojem,
ale jsou samy o~sobě prostředkem komunikace. Stejně jako volba obrázku
nebo~formulace vytváří typografie spolu s~obsahem
celkové sdělení zprávy \parencite{Pilka:2019:CharakterPisma}. 
Jak to? V~roce 1993 provedli Poffenberger a~Barrows experiment,
ve~kterém dokázali, že jednoduché čáry mohou v~člověku vyvolat určité emoce.
Pro~mozek je pohyb oka při sledování linie podobný fyzickému pohybu,
což evokuje samotnou emoci, např.~klesající linie evokují smutek nebo~slabost,
zatímco stoupající linie jsou spojeny s~hravostí nebo~radostí
\parencite{Sladovnikova:2021:EmoceTextu}.

Existuje mnoho typů písem\,---\,fontů\,---\,z~nichž se~každé hodí k~něčemu
jinému: tvrdé a~technické grotesky pro~snadné čtení, seriózní antikvy
pro~technické dokumenty nebo~vlídná humanistická písma pro~méně formální situace
\parencite{Samara:2010:ZakladyDesignu}.
Kromě vhodnosti může náš vkus v~psaní leccos vypovídat o~naší osobnosti.
Podle~průzkumu společnosti BlueGhost dávají techničtější lidé přednost ostřejším
písmům, jako je Times Roman, uvolněnější lidé dávají přednost jemným písmům,
jako je Montserrat nebo~Helvetica, legračním lidem se líbí Comic Sans
a~romantičtější a~emocionálnější lidé dávají nejvíce přednost kaligrafickým
písmům \parencite{Pilka:2019:CharakterPisma}.

\section*{Kaligrafické písmo}

Právě kaligrafické písmo (z~řečtiny doslova~\emph{krásné písmo}
neboli~\emph{krasopis}) zvolil nově zvolený český prezident Petr\,Pavel
pro~svůj slavnostní slib, který podepsal \formatdate{9}{3}{2023}
při~své inauguraci. Jedná se o~výraznou odlišnost oproti přísahám
například Miloše\,Zemana (Arial a Times Roman),
Václava\,Klause (Arial a Times Roman) nebo~Václava\,Havla (Arial).
Tento slib je ručně psaný typografkou Petrou\,Dočekalovou
\parencite{Font:2023:SlibPrezidenta}.
Použitím tohoto písma vypadá slib nejen elegantněji, ale~také osobněji
a~lidštěji\,---\,kaligrafické písmo se~proto používá také
na~pozvánky, pamětní listiny nebo~osobnější oznámení
\parencite{Kaucky:2018:Kaligrafie}.
Přestože ruční písmo od~zavedení knihtisku v~15.\ století ustupuje,
kaligrafie se~kvůli své osobní a~umělecké povaze stále používá jako
okrasné písmo \parencite{Zapf:2007:Alphabet}.
Přesto se~forma slibu setkala s~kritikou a~posměšky ze~strany lidí
na~internetu, zatímco kritici hodnotili formu slibu pozitivně,
podle~zpravodajského webu \textcite{Seznam:2023:KritikaSlibu}.

Kaligrafické písmo samo o~sobě není vhodné pro~všechny situace.
Časté změny písma nebo~použití nevhodného fontu mohou čtenáře zahltit
a~působit rušivě \parencite{Rajlich:1998:TypografiePrelomu}.
Přesto si kaligrafie v~dnešní společnosti své místo nachází.
Moderní kaligrafie se dodnes používá pro~budování stylu,
faksimile (napodobeniny) starých dokumentů pro~muzea a~knihovny,
grafický design (reklamy, vizuální identita) nebo~jako samostatná umělecká
forma \parencite{Kafka:2018:Kaligrafie}.

\newpage
\renewcommand{\refname}{Literatura}
\printbibliography{}

\end{document}